\section{System Description}

The device studied throughout this project is a linear system with a static nonlinear feedback loop. It has a single input and a single output and the feedback is known to follow the input-output relation given by
\begin{equation}
    y(t) = u^3(t).
    \label{eq:NL_feedback}
\end{equation}

The objective will be to identify the linear part of the system and to have a measure of the nonlinearity and noise levels, which is done in Sec.~\ref{sec:BLA} and to then use different non linear models to describe the system in Sec.~\ref{sec:NL_models}.

The measurement device (referred to as the \textit{PXI setup}) allows to send only multisines and their RMS value cannot exceed $1$~V. Its ADCs have a range that can be chosen and can sample at high frequencies ($> 1$~GHz).

This report deliberately focusses on the theoretical points and the results rather than the implementation because the Matlab™ scripts and the measured data are available on GitHub\footnote{https://github.com/e-colot/Nonlinear-identification}. 

Note that this report is the continuation of one that focussed only on linear modelling\footnote{https://github.com/e-colot/LTI-identification} and is therefore not complete about the subject.






\section{Best Linear Approximation (BLA)}
\label{sec:BLA}

A multisine has been applied to the system for multiple periods. This was then repeated over multiple realizations were the phase of the multisine was randomly generated each time.

The robust method is applied to the measured signals, which can be summarized as follows: Averaging over the periods allows to remove noise and the non linear distortions are removed by averaging over the realizations.

The variances are then computed over the realizations and Fig.~\ref{fig:FRF} shows the results.

\begin{figure} [h!]
    \centering
    \includegraphics[width=0.48\textwidth]{Pictures/FRF.png}
    \caption{BLA and standard deviation of non linear distortions and noise.}
    \label{fig:FRF}
\end{figure}


\vfill\eject % column break
\section{Non Linear Models}
\label{sec:NL_models}

\subsection{Memory Polynomial Model}

The first NL model to be applied does not take into consideration the knowledge of~(\ref{eq:NL_feedback}). A simple model is the 



